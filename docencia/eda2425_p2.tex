\documentclass[10pt,a4paper]{../documents/class_material_assignatura_udg}

\usepackage{wrapfig}
\usepackage{xspace}
\newcommand{\Cpp}{{C\texttt{++}}\xspace}

\usepackage{listings}
\usepackage[shortlabels]{enumitem}
\usepackage{tikzsymbols} % pel \Ninja

\lstdefinestyle{codicpp}{
  language=C++,
  basicstyle=\ttfamily\scriptsize,%\color{black!75}\ttfamily\scriptsize,
  %showstringspaces=false,
  backgroundcolor=\color{gris!50},
  keywordstyle=\color{blaufosc}\ttfamily,
  stringstyle=\color{salmo}\ttfamily,
  commentstyle=\color{grocintens!85!black}\ttfamily,
  morekeywords={cout,cin,string },
  morecomment=[l][\color{verd}]{\#},
  upquote=true,%per ' i \" a l'SQL
  escapechar=¿,
  moredelim=**[is][\btHL]{@}{@},
  moredelim=**[is][{\btHL[fill=grocsuau]}]{`}{`},
}


\lstdefinestyle{codibash}{
  language=bash,
  basicstyle=\ttfamily,
  showstringspaces=false,
  backgroundcolor=\color{gris},
  commentstyle=\color{salmo},
  keywordstyle=\color{blaufosc},
  otherkeywords={$},
  escapechar=¡,
  moredelim=**[is][\btHL]{@}{@},
  moredelim=**[is][{\btHL[fill=grocsuau]}]{`}{`},
}




%%%%%%%%%%%%%%%%%%%%%%%%%%%%%%%%%%%%%% INFO %%%%%%%%%%%%%%%%%%%%%%%%%%%%%%%%%%%%%%
\graphicspath{{../logos/}{./imatges/}}% Lloc on hi ha els logos
%        { \huge \bfseries \sc{Pràctica 1: Estructures de Dades} \linebreak \linebreak Juguem amb Ce Essa Ves? \linebreak\linebreak 

\title{Pràctica 2: Horaris d'examens}
\shorttitle{Pràctica 2: Algorítmica}
\author{Jerónimo Hernández}
\email{jeronimo.hernandez@udg.edu}
\assignatura[EDA]{Estructures de Dades i Algorítmica}
\curs[24/25]{Curs 2024/25}
\grau[GEINF - GDDV]{Grau en Enginyeria Informàtica i Grau en Disseny i Desenvolupament de Videojocs}
% Comprovar els "ACTUALITZAR DATA"
%%%%%%%%%%%%%%%%%%%%%%%%%%%%%%%%%%%%%%%%%%%%%%%%%%%%%%%%%%%%%%%%%%%%%%%%%%%%%%%%%%


\begin{document}
\maketitle
\tableofcontents
\pagestyle{fancy}


%\documentclass[10pt,a4paper]{article}
%\usepackage{graphicx}
%\graphicspath{{../imatges/}{./imatges}}% Lloc on hi ha els logos
%
%\usepackage[catalan]{babel}
%\usepackage[utf8]{inputenc}
%\usepackage[margin=2.5cm]{geometry}
%\usepackage{graphicx}
%\usepackage[x11names,table]{xcolor}
%\usepackage{scalerel} % per escalar gràfics a mida lletres
%\usepackage{hyperref}
%\usepackage{nameref}
%\usepackage{listings} % per codi
%\usepackage{comment} % per amagar troços
%\usepackage{lscape} % per posar pàgines apaisades
%\usepackage{fancyhdr}
%\usepackage{fancyvrb}
%\usepackage{lstautogobble} % per listings amb indentació
%\usepackage{makecell}
%\usepackage{multicol}
%\usepackage{wrapfig}
%\usepackage{ulem} % per \sout, tatxar text
%\usepackage{fontawesome} % per fletxes... 
%\usepackage{array} % per centrar taules verticalment
%\usepackage{tikzsymbols} % pel \Ninja
%\usepackage{amsmath} % per equacions
%
%% Colors UdG
%\definecolor{blauFosc}{RGB}{0,20,137}% reflexBlueC
%\definecolor{verd}{RGB}{0,115,119}% 3278C
%\definecolor{salmo}{RGB}{229,106,84}% 7416C
%\definecolor{daurat}{RGB}{246,190,0}% 7408C
%\definecolor{gris}{RGB}{217,217,214}% CoolGRay1C
%
%% Colors terreny (Pràctica 2)
%\definecolor{mar}{RGB}{0,150,200}
%\definecolor{terra}{RGB}{150,100,50}
%\definecolor{lava}{RGB}{160,0,80}
%\definecolor{edificis}{RGB}{255,200,0}
%
%% Nou tipus de columna per centrar horitzonalment i vertical
%\newcolumntype{C}[1]{>{\centering\arraybackslash}m{#1}}
%
%\usepackage{lipsum}
%
%\newcommand{\Cpp}{{C\texttt{++}}}
%
%\pagestyle{fancy}
%\lstset{language=C++,basicstyle=\ttfamily\scriptsize,escapechar={¿},numbers=left,tabsize=4}
%\lstset{upquote=true} %per ' i \" als programes
%
%\lhead{EDA \sc{Pràctica 2: Algorísmica}, curs 2023/24 GDDV/GEINF}
%\rhead{\thepage}
%\cfoot{}
%\begin{document}
%    \begin{titlepage}
%        \newcommand{\HRule}{\rule{\linewidth}{0.5mm}} % Defines a new command for the horizontal lines, change thickness here
%        \begin{flushleft}
%            \includegraphics[height=1.5cm]{EPS.png}\\\vfill
%        \end{flushleft}
%        \center % Center everything on the page
%        %----------------------------------------------------------------------------------------
%        %	HEADING SECTIONS
%        %----------------------------------------------------------------------------------------
%        \textsc{\huge \bfseries Pràctiques EDA}\\[0.25cm]
%        \textsc{\Large \bfseries Curs 2023/24}\\[0.25cm]
%        \textsc{\large GEINF - GDDV}
%        %----------------------------------------------------------------------------------------
%        %	TITLE SECTION
%        %----------------------------------------------------------------------------------------
%        \HRule \\[0.4cm]
%        { \huge \bfseries \sc{Pràctica 2: Algorísmica} \linebreak \linebreak Auditant la Gencat}\\[0.4cm] % Title of your document
%        \HRule \\\vfill
%        \begin{minipage}{0.4\textwidth}
%            \begin{flushleft}
%                \includegraphics[height=1.5cm]{logo_IMAE.png}
%            \end{flushleft}
%        \end{minipage}
%        \hfill
%        \begin{minipage}{0.4\textwidth}
%            \begin{flushright} \large
%                {\small (PDF generat el \today)}
%            \end{flushright}
%        \end{minipage}
%    \end{titlepage}



\section{Presentació}
\label{sec:presentacio}

%\begin{wrapfigure}{r}{0.36\textwidth}
%\centering
%    \includegraphics[width=.7\linewidth]{logo.png}
%\end{wrapfigure}

L'EPS de la UdG ens ha demanat un programa per crear els horaris d'examens finals de cada semestre. S'ha de tenir en compte que:

\begin{itemize}
\item A l'EPS s'imparteixen assignatures de diferents graus.
\item Cada semestre, hi ha examens finals de diferents assignatures sota petició del professorat responsable.
\item Hi ha dos tipus d'assignatura segons el nombre de matriculats: assignatures de {\bf\textcolor{salmo}{g}ran} grup, i de grup {\bf\textcolor{salmo}{r}eduït}.
\item L'EPS disposa d'un nombre limitat d'aules per als examens.
\item Hi ha aules de l'EPS on no hi quep tot l'alumnat de les assignatures de {\bf\textcolor{salmo}{g}ran} grup.
\item Hi haurà diferents torns d'examens, i cada torn es disposarà d'un nombre fix d'aules de {\bf\textcolor{salmo}{g}ran} capacitat, i de capacitat {\bf\textcolor{salmo}{r}eduïda}.
\item L'EPS fixarà, si s'escau, el nombre màxim de dies d'examen.
\item L'EPS prohibeix dos examens del mateix curs i grau en el mateix torn. També es reserva el dret de demanar que els examens de dues assignatures no coincideixin (p.ex., tenen el mateix professor).\\[-2pt]

\item Tot i la possible restricció d'un màxim de dies, es vol fer servir el menor nombre de dies possible.
\item Tot i la condició de prohibir examens del mateix curs i grau en el mateix torn, es vol que estiguin el màxim possible de repartits al llarg dels torns (dies).  
\end{itemize}

Ens demanen un programa que faci l'assignació d'examens (assignatures) a dies. Hem comunicat a l'EPS que, donat la complexitat del problema, la solució òptima pot tenir un alt cost computacional. L'EPS ens demana la \textbf{possibilitat de triar} cercar la solució ràpida, vàlida, o l'òptima.

\section{Feina a fer}

\begin{enumerate}
    \item Fent servir una estratègia voraç, \textbf{trobar ràpidament} una assignació d'assignatures a torns que no superi el nombre màxim donat (si s'indica) i que compleixi les restriccions establertes. Si no es troba, s'informarà que no ha pogut trobar una solució.

    \item Fent servir \textit{backtracking}, \textbf{trobar una} assignació d'assignatures a torns que no superi el nombre màxim donat (si s'indica) i que compleixi les restriccions establertes. Si no és possible trobar una assignació amb aquestes condicions, s'informarà que no hi ha solució.
    
    \item Fent servir \textit{backtracking}, \textbf{trobar la millor} assignació d'assignatures a torns, és a dir, la que complint totes restriccions donades, \textit{minimitza} el nombre de torns, primer, i maximitza la desviació estàndard mitjana dels torns als que s'han assignat examens dels mateix curs-grau. Igualment, si no és possible trobar una assignació amb aquestes condicions, s'informarà que no hi ha solució.
\end{enumerate}

\subsection{Característiques del programa}\label{sec:caracteristiques}

Així, heu de fer un programa que:

\begin{enumerate}
\item Rebi \textbf{per paràmetre} des de la terminal els valors que configuren el problema:
\begin{itemize}
\item[\texttt{-cr <int>}]: nombre d'aules de capacitat {\bf\textcolor{salmo}{r}eduïda}. Valor per defecte, 1.
\item[\texttt{-gc <int>}]: nombre d'aules de {\bf\textcolor{salmo}{g}ran} capacitat. Valor per defecte, 1.
\item[\texttt{-d <int>}]: nombre de dies màxim en que poden haver examens. Valor per defecte, \textbf{indeterminat}.
\item[\texttt{-s <int>}]: semestre amb el que es vol treballar. Valor per defecte, 1.
\item[\texttt{fitxer\_dades}]: el nom (i ruta) del fitxer de dades que es vol fer servir
\end{itemize}

\hspace{-15pt}1-bis. Rebi \textbf{també per paràmetre} des de la terminal quin algoritme vol executar l'usuari:
\item[``'']: per defecte (no caldrà passar cap opció), s'executarà l'algoritme de \textit{backtracking} que torna \textbf{una solució vàlida} (la primera).
\item[\texttt{-m}]: s'executarà l'algoritme de \textit{backtracking} que torna la \textbf{millor solució}.
\item[\texttt{-v}]: s'executarà l'algoritme que \textbf{cerca ràpidament} una solució de manera voraç.
\item Llegeixi les \textbf{assignatures} del fitxer de tipus text indicat. Aquest fitxer també podrà contenir \textbf{restriccions} entre parells d'assignatures. Vegeu-ne un exemple a la Secció~\ref{sec:entrada}.

%Només \textbf{conservarem les dades del semestre amb el que volem treballar}. 
Podeu fer els preprocessaments de les dades que necessiteu.

\item Executi l'algoritme indicat per l'usuari.
\item Mostri per pantalla la \textbf{distribució d'assignatures} (examens) en franjes horàries proposta. 
També mostrareu el \textbf{temps de còmput} necessari per trobar la solució, el nombre total de torns i de dies, i la \textbf{desviació estàndar mitjana}.

Vegeu-ne varis exemples a la Secció~\ref{sec:sortida}.
\end{enumerate}

\subsection{Plantejament del problema}

\begin{itemize}
\item Hi ha disponibles un nombre determinat d'aules de {\bf\textcolor{salmo}{g}ran} capacitat, i de capacitat {\bf\textcolor{salmo}{r}eduïda}, indicats per paràmetre.

%Una assignatura de grup {\bf\textcolor{salmo}{g}ran} només pot anar a una aula de {\bf\textcolor{salmo}{g}ran} capacitat. 
\item \textbf{Cada dia} hi poden haver \textbf{dos torns} (matí i tarda) d'examens.
\item S'han de repartir \textbf{totes les assignatures} del semestre indicat per paràmetre.
\item Cada torn pot incloure \textbf{tants examens com aules} disponibles respectant la restricció de capacitat de les aules:

Una assignatura de grup {\bf\textcolor{salmo}{r}eduït} podria ocupar un aula de {\bf\textcolor{salmo}{g}ran} capacitat, però una assignatura de {\bf\textcolor{salmo}{g}ran} grup mai pot anar a un aula de capacitat {\bf\textcolor{salmo}{r}eduïda}.
\item S'han de complir les \textbf{restriccions entre assignatures}:
    \begin{itemize}
        \item Dos assignatures del mateix grau i curs no es poden assignar al mateix torn.
        \item Restricció específica entre qualsevol parell d'assignatures que s'indiqui al fitxer de dades.
    \end{itemize}

\item Es poden fer servir, en principi, tants torns com calgui, però...
    \begin{itemize}
        \item es pot indicar el nombre de dies màxim del període d'examens, 
        
        \item o es pot demanar buscar la solució que faci servir el \textbf{mínim nombre de torns possible}.
    \end{itemize}

\item En triar entre dos solucions amb el mateix nombre de torns, es prefereix aquella que reparteixi millor les assignatures del mateix curs i grau al llarg dels \textbf{torns}.

En el següent esquema, la solució \texttt{b} és millor que l'\texttt{a}.

\texttt{\ \ \ \ T-1 T-1 T-2 T-2 T-3 T-3 T-4 T-4} \hfill\texttt{// T-i} és l'\texttt{i}-èsim torn

\texttt{a. I1a I2a I1b I2b V1a V2a V1b V2b} \hfill\texttt{// I1a} és l'assignatura \texttt{a} del curs \texttt{1} del grau \texttt{I}

\texttt{b. I1a I2a V1a V2a I1b I2b V1b V2b}

Es farà servir la \textbf{desviació estàndard mitjana} per comparar el repartiment d'assignatures en torns. Per cada grau i curs, es calcula la desviació estàndard dels torns als que s'assignen assignatures d'aquest grau-curs. 

        \begin{center}
    \begin{equation}
        \begin{tabular}{m{4cm}m{0cm}m{8.5cm}}
            \[
                \sigma = \sqrt{\frac{1}{n} \sum_{i=1}^n (t_i - \overline{t})^2}
            \]
            & &
            Fórmula de la desviació estàndard $\sigma$, on $n$ és el nombre d'assignatures del grau-curs, $t_{i}$ és el torn al que s'ha assignat l'$i$-èsima assignatura, i $\overline{t}$ és el torn mitjà entre els que s'han fet servir per al grau-curs.
        \end{tabular}
        \label{eq:desviacio}
    \end{equation}
\end{center}

Després es calcula la mitjana entre tots els graus i cursos. Una solució és \textbf{millor} si la desviació estándard mitjana és \textbf{més gran}.
\end{itemize}


        




\subsection{Instruccions}

\begin{itemize}
    \item Cal codificar el \textit{backtracking} utilitzant les classes vistes a teoria: \texttt{Solucionador}, \texttt{Solucio} i \texttt{Candidats}.
    
    \item Podeu codificar l'algorisme voraç com creieu més convenient.
    
    \item Per llegir les dades podeu fer servir les funcions d'\texttt{eines.h}.
    
    \item Ens quedarem amb les assignatures del semestre indicat. Les altres es poden ignorar.

    \item Si es creu oportú, es poden preprocessar les dades per reorganitzar-les o construir estructures auxiliars que puguin millorar l'eficiència dels algoritmes. 
    
    Tot preprocessament s'inclourà en el càlcul del \textbf{temps de còmput} de l'algorisme.

    \item En una execució normal, l'únic paràmetre imprescindible és el nom del \textit{fitxer de dades}. Cal fer servir els valors per defecte indicats a la Secció~\ref{sec:caracteristiques}. 
    
    S'ha de donar a l'usuari una opció per consultar com executar el programa (ajuda, \texttt{-h}).

    \item Mostrar per sortida estàndard el resultat en un format similar al que es mostra a la Secció~\ref{sec:sortida}.
    
    
    
    \item Per fer les proves, cal començar amb poques línies amb imports ben dimensionats (pel que fa al màxim que se li podrà assignar a un auditor), i anar-ho incrementant de mica en mica (el temps d'execució es dispararà!).
\end{itemize}

\clearpage

\section{Instruccions de lliurament}

Caldrà penjar a \textbf{Moodle} un fitxer comprimit \textbf{\texttt{zip}} (o bé \textbf{\texttt{tgz}}) amb un nom que segueixi el format

\begin{center}
	\texttt{CodiUsuari\_CognomsNom\_p2.zip}
\end{center}

\noindent que ha de contenir:

\begin{enumerate}

	\item El codi font.
	
	A part del correcte funcionament, es valorarà la documentació i la claredat del codi.

	\item Un joc de proves (fitxers amb les dades utilitzades).

	\item Un \textit{script} de \textit{bash} que permeti executar el vostre programa sobre el joc de proves (executant una prova darrere l'altra, sense preguntar res).

	\item Un fitxer \texttt{llegeix.me} amb informació bàsica del programa i aclariments que puguin facilitar la correcció (p.e., si alguna cosa no funciona bé).
	
	\item\label{informe} Un document anomenat \texttt{informe.pdf} amb una explicació de l'estratègia que heu seguit en les vostres implementacions. També mostreu les proves més significatives que heu realitzat. Per cada prova hi constarà:
	
	\begin{itemize}
	    \item Comanda per executar-la (p.e., \texttt{./p2 -m -d 3 assignatures\_poques.txt}).
	    \item Resultat obtingut (assignació dels examens, desviació estàndard mitjana i temps).
	    \item Anàlisi: què buscàveu amb aquesta prova, com ara ``comprovar si es tenien en compte les restriccions'', ``comprovar com es dispara el temps d'execució'', ``veure si es detectava bé que no hi havia solució'', ``comprovar el pas de paràmetres al \texttt{main}'', etc.
	\end{itemize}
	 L'informe tindrà una pàgina de portada amb títol, nom, etc. Es valorarà que estigui ben escrit.

	\item\label{doxygen} Una carpeta \texttt{html} amb la documentació generada amb {\sf doxygen} en format HTML.

    Inclourà la descripció de cada classe, la dels atributs, i les pre- i postcondicions dels mètodes. S'ha de mostrar aquesta informació tant per la \textbf{part pública} com per la \textbf{part privada} de les classes.
    
    \textbf{Compte:} per defecte, {\sf doxygen} només mostra la part pública a la documentació. Per aconseguir-ho, heu de marcar la casella \texttt{EXTRACT\_PRIVATE} a la pestanya \texttt{Expert} de {\bf doxywizard}.

\end{enumerate}

\bigskip

\noindent A \textbf{\texttt{bas.udg.edu}} cal deixar, al directori \texttt{$\sim$/eda/p2}, els fitxers de codi, joc de proves, \textit{script} i \texttt{llegeix.me} (punts 1-4). No hi pengeu la documentació (punts~\ref{informe} i~\ref{doxygen}); ocupa massa i no ens cal aquí.

\bigskip

Recordeu que:

\begin{itemize}
    \item Tots els fitxers de tipus text han d'estar en format Unix.
    
    \item Durant el procés de correcció es podrà demanar una entrevista amb l'estudiant per aclarir aspectes de la pràctica.
\end{itemize}

Consulteu el document \textit{``Instruccions per al lliurament de les activitats de laboratori''} per més detalls.

\bigskip

\textbf{La data límit de lliurament està especificada a l'activitat corresponent de Moodle.}

\clearpage


\section{Joc de proves}

Cal que acompanyeu el vostre codi amb un joc de proves. En aquest cas, el joc de proves es redueix a diferents fitxers de dades. Com sempre, us donem un joc de proves bàsic que \textbf{heu de complementar}.

El joc de proves que compartim consta de 2 fitxers de dades (\textcolor{salmo}{\bf no son els mateixos que als exercicis}) i el teniu disponible a Moodle. Aquí us mostrem diferents sortides possibles davant diferents crides. Vigileu perquè segons com es tracta el problema pels vostres algoritmes voraç i de {\it backtracking}, la sortida que obteniu pot ser diferent a la que us donem. En el cas de l'algoritme voraç i el de \textit{backtracking} d'una solució qualsevol, la sortida pot ser molt diferent. En el cas de l'algoritme de \textit{backtracking} que retorna la millor solució, aquesta també pot ser diferent a la que us donem però ha de ser equivalent d'acord amb el criteri d'optimització (mateix nombre de torns, mateixa desviació estàndard mitjana). 

A continuació reproduïm el contingut de \texttt{assignatures\_poques.txt} i les sortides del programa davant de diferent crides.



\subsection{Fitxer de dades: \texttt{assignatures\_poques.txt}}\label{sec:entrada}
\begin{verbatim}
Grau	Assignatura	Codi	Tipus	Crèdits	Semestre	Curs
GDDV	Elements matemàtics per a videojocs	3105G13001	r	9.00	1	1
GDDV	Arquitectures de consoles i dispositius de videojocs	3105G13009	r	7.00	2	1
GDDV	Formació d'imatges i interacció entre objectes	3105G13010	r	6.00	2	1
GDDV	Art i videojocs	3105G13031	r	3.00	1	1
GDDV	Disseny conceptual dels videojocs	3105G13027	r	5.00	2	2
GDDV	Narrativa dels videojocs	3105G13029	r	7.00	1	2
GDDV	Disseny de motors de jocs I	3105G13022	r	5.00	1	3
GDDV	Disseny de motors de jocs II	3105G13023	r	4.00	2	3
GDDV	Cloud computing i sistemes distribuïts per a videojocs	3105G13026	r	5.00	2	3
GDDV	Teoria i pràctica de la producció audiovisual	3105G13030	r	5.00	1	3
GEINF	Àlgebra	3105G07001	g	6.00	1	1
GEINF	Càlcul	3105G07002	g	6.00	1	1
GEINF	Lògica i matemàtica discreta	3105G07003	g	9.00	2	1
GEINF	Metodologia i tecnologia de la programació II	3105G07009	g	6.00	2	1
GEINF	Estadística	3105G07007	r	6.00	1	2
GEINF	Organització i administració d'empreses	3105G07008	g	6.00	2	2
GEINF	Estructures de dades i algorítmica	3105G07010	g	9.00	1	2
GEINF	Projecte de programació	3105G07011	r	5.00	2	2
GEINF	Multimèdia i interfícies d'usuari	3105G07012	g	5.00	1	3
GEINF	Fonaments de computació	3105G07013	r	5.00	1	3
GEINF	Paradigmes i llenguatges de programació	3105G07014	r	5.00	2	3
GEINF	Intel·ligència artificial	3105G07015	g	5.00	2	3
GEB	Fonaments de física 1	3105G00001	g	6.00	1	1
GEB	Fonaments de física 2	3105G00002	g	6.00	2	1
GEB	Fonaments de matemàtiques 2	3105G00003	g	6.00	2	1
GEB	Fonaments de matemàtiques 1	3105G00004	g	9.00	1	1
GEB	Estadística	3105G00015	g	6.00	1	2
GEB	Teoria de circuits	3105G03012	g	6.00	1	2
GEB	Electrònica analògica	3105G03022	g	6.00	2	2
GEB	Anatomia funcional i biomecànica	3109G01057	r	5.00	2	2
GEB	Desenvolupament de projectes d'electrònica	3105G03032	g	4.00	2	3
GEB	Disseny de dispositius d'assistència i teràpia	3105G15007	r	5.00	1	3
GEB	Equips de monitorització i diagnòstic	3105G15008	r	6.00	1	3
GEB	Gestió intel·ligent de dades i coneixement mèdic	3105G15009	r	5.00	1	3
*
3105G07001	3105G13001
3105G07002	3105G13001
3105G07001	3105G13031
3105G07002	3105G13031
3105G07003	3105G13009
3105G07009	3105G13009
\end{verbatim}


On l'asterisc `\texttt{*}' és el símbol que es fa servir per separar el llistat d'assignatures de les restriccions entre assignatures (cada línia representa una restricció entre un parell d'assignatures --no poden ser assignades al mateix torn--).

\subsection{Funcionament bàsic}

\begin{lstlisting}[style=codibash]
$ ./p2
Falten arguments ("p2 --help" per ajuda)
\end{lstlisting}

\begin{lstlisting}[style=codibash]
$ ./p2 --help
¡Ú¡s: ./p2 [-h] | [-v] [-m] [-cr <int>] [-gc <int>] [-s <int>] [-d <int>] fitxer

opcio pot ser:
  -h, --help     mostra aquest missatge d'ajuda i surt

  -v             cerca r¡à¡pida amb un algoritme vora¡ç¡
  -m             cerca la soluci¡ó¡ que minimitza el nombre de torns i 
                   maximitza la dispersi¡ó¡
  -cr <int>      assigna <int> com a nombre d'aules de capacitat
                   redu¡ï¡da disponibles per als examens
  -gc <int>      assigna <int> com a nombre d'aules de gran
                   capacitat disponibles per als examens
  -s <int>       indica que s'ha de fer l'assignaci¡ó¡ per al
                   semestre <int> (1 o 2)
  -d <int>       indica el l¡í¡mit m¡à¡xim de dies que es poden fer servir

fitxer           fitxer de text amb totes les assignatures a les
                   que es vol assignar un torn d'examen i possibles
                   restriccions entre parells d'assignatures
\end{lstlisting}

\begin{lstlisting}[style=codibash]
$ ./p2 -s -cr 3
Error: El valor associat a l'opci¡ó¡ '-s' ¡é¡s incorrecte.
\end{lstlisting}

%\begin{lstlisting}[style=codibash]
%$ ./p2 -s 2
%Error: Falta el nom del fitxer.
%\end{lstlisting}


\begin{lstlisting}[style=codibash]
$ ./p2 -s 2
Error: Falta el nom del fitxer.
\end{lstlisting}

\begin{lstlisting}[style=codibash]
$ ./p2 -s 2 noexisteix.txt
Error: El fitxer [noexisteix.txt] no es pot obrir. Repassa el nom i permisos.
\end{lstlisting}

\subsection{Funcionament}\label{sec:sortida}

Busquem l'assignació amb un algoritme voraç:
\begin{lstlisting}[style=codibash]
$ ./p2 -v -cr 3 -gc 2 assignatures_poques.txt
18 assignatures leides de 3 graus diferents.

*********************************
* Torn 1                   n=5  *
*-------------------------------*
* 3105G00001 (tipus g),   GEB-1 *
* 3105G00015 (tipus g),   GEB-2 *
* 3105G13001 (tipus r),  GDDV-1 *
* 3105G13022 (tipus r),  GDDV-3 *
* 3105G07013 (tipus r), GEINF-3 *
*********************************

*********************************
* Torn 2                   n=4  *
*-------------------------------*
* 3105G07002 (tipus g), GEINF-1 *
* 3105G00004 (tipus g),   GEB-1 *
* 3105G15007 (tipus r),   GEB-3 *
* 3105G07007 (tipus r), GEINF-2 *
*********************************

*********************************
* Torn 3                   n=4  *
*-------------------------------*
* 3105G03012 (tipus g),   GEB-2 *
* 3105G07001 (tipus g), GEINF-1 *
* 3105G13029 (tipus r),  GDDV-2 *
* 3105G15009 (tipus r),   GEB-3 *
*********************************

*********************************
* Torn 4                   n=5  *
*-------------------------------*
* 3105G07010 (tipus g), GEINF-2 *
* 3105G07012 (tipus g), GEINF-3 *
* 3105G15008 (tipus r),   GEB-3 *
* 3105G13031 (tipus r),  GDDV-1 *
* 3105G13030 (tipus r),  GDDV-3 *
*********************************

Num. torn: 4
Num. dies:  2
Desviaci¡ó¡: 0.924055
Temps: 0.00129841 segons
\end{lstlisting}
o amb \textit{backtracking}, però la versió que es queda amb la primera solució (opció per defecte del programa):

\begin{lstlisting}[style=codibash]
$ ./p2 -cr 3 -gc 2 assignatures_poques.txt
18 assignatures leides de 3 graus diferents.

*********************************
* Torn 1                   n=5  *
*-------------------------------*
* 3105G00001 (tipus g),   GEB-1 *
* 3105G00015 (tipus g),   GEB-2 *
* 3105G13001 (tipus r),  GDDV-1 *
* 3105G15007 (tipus r),   GEB-3 *
* 3105G13029 (tipus r),  GDDV-2 *
*********************************

*********************************
* Torn 2                   n=5  *
*-------------------------------*
* 3105G07002 (tipus g), GEINF-1 *
* 3105G00004 (tipus g),   GEB-1 *
* 3105G15008 (tipus r),   GEB-3 *
* 3105G13022 (tipus r),  GDDV-3 *
* 3105G07007 (tipus r), GEINF-2 *
*********************************

*********************************
* Torn 3                   n=5  *
*-------------------------------*
* 3105G03012 (tipus g),   GEB-2 *
* 3105G07001 (tipus g), GEINF-1 *
* 3105G15009 (tipus r),   GEB-3 *
* 3105G07013 (tipus r), GEINF-3 *
* 3105G13030 (tipus r),  GDDV-3 *
*********************************

*********************************
* Torn 4                   n=3  *
*-------------------------------*
* 3105G07010 (tipus g), GEINF-2 *
* 3105G07012 (tipus g), GEINF-3 *
* 3105G13031 (tipus r),  GDDV-1 *
*********************************

Num. torns: 4
Num. dies:  2
Desviaci¡ó¡: 0.701833
Temps: 0.000919688 segons
\end{lstlisting}

O busquem la millor solució amb \textit{backtracking}:

\begin{lstlisting}[style=codibash]
$ ./p2 -m -cr 3 -gc 2 assignatures_poques.txt
18 assignatures leides de 3 graus diferents.

*********************************
* Torn 1                   n=5  *
*-------------------------------*
* 3105G00001 (tipus g),   GEB-1 *
* 3105G00015 (tipus g),   GEB-2 *
* 3105G13001 (tipus r),  GDDV-1 *
* 3105G13022 (tipus r),  GDDV-3 *
* 3105G07013 (tipus r), GEINF-3 *
*********************************

*********************************
* Torn 2                   n=4  *
*-------------------------------*
* 3105G07002 (tipus g), GEINF-1 *
* 3105G07010 (tipus g), GEINF-2 *
* 3105G15007 (tipus r),   GEB-3 *
* 3105G13029 (tipus r),  GDDV-2 *
*********************************

*********************************
* Torn 3                   n=4  *
*-------------------------------*
* 3105G00004 (tipus g),   GEB-1 *
* 3105G03012 (tipus g),   GEB-2 *
* 3105G15008 (tipus r),   GEB-3 *
* 3105G13031 (tipus r),  GDDV-1 *
*********************************

*********************************
* Torn 4                   n=5  *
*-------------------------------*
* 3105G07001 (tipus g), GEINF-1 *
* 3105G07012 (tipus g), GEINF-3 *
* 3105G07007 (tipus r), GEINF-2 *
* 3105G15009 (tipus r),   GEB-3 *
* 3105G13030 (tipus r),  GDDV-3 *
*********************************

Num. torns: 4
Num. dies:  2
Desviaci¡ó¡: 0.979611
Temps: @1.52839 segons@
\end{lstlisting}

Si només reduïm el nombre d'aules disponibles (\texttt{-cr 2}), el temps es comença a disparar:

\begin{lstlisting}[style=codibash]
$ ./p2 -m -cr 2 -gc 2 assignatures_poques.txt
18 assignatures leides de 3 graus diferents.

*********************************
* Torn 1                   n=4  *
*-------------------------------*
* 3105G00001 (tipus g),   GEB-1 *
* 3105G00015 (tipus g),   GEB-2 *
* 3105G13001 (tipus r),  GDDV-1 *
* 3105G13022 (tipus r),  GDDV-3 *
*********************************

*********************************
* Torn 2                   n=4  *
*-------------------------------*
* 3105G07002 (tipus g), GEINF-1 *
* 3105G07010 (tipus g), GEINF-2 *
* 3105G15007 (tipus r),   GEB-3 *
* 3105G07013 (tipus r), GEINF-3 *
*********************************

*********************************
* Torn 3                   n=3  *
*-------------------------------*
* 3105G00004 (tipus g),   GEB-1 *
* 3105G13029 (tipus r),  GDDV-2 *
* 3105G15008 (tipus r),   GEB-3 *
*********************************

*********************************
* Torn 4                   n=3  *
*-------------------------------*
* 3105G03012 (tipus g),   GEB-2 *
* 3105G07001 (tipus g), GEINF-1 *
* 3105G15009 (tipus r),   GEB-3 *
*********************************

*********************************
* Torn 5                   n=4  *
*-------------------------------*
* 3105G07012 (tipus g), GEINF-3 *
* 3105G13031 (tipus r),  GDDV-1 *
* 3105G07007 (tipus r), GEINF-2 *
* 3105G13030 (tipus r),  GDDV-3 *
*********************************

Num. torns: 5
Num. dies:  3
Desviaci¡ó¡: 1.25739
Temps: @38.9267 segons@
\end{lstlisting}

Noteu que, degut a les restriccions, no hi ha solució possible si volem limitar-ho a 2 dies:
\begin{lstlisting}[style=codibash]
$ ./p2 -m -cr 2 -gc 2 -d 2 assignatures_poques.txt
18 assignatures leides de 3 graus diferents.

No hi ha soluci¡ó¡ per la configuraci¡ó¡ indicada
Temps: 1.20769 segons
\end{lstlisting}

Amb més assignatures, la cerca d'una solució amb \textit{backtracking} dona:
\begin{lstlisting}[style=codibash]
$ ./p2 -cr 3 -gc 2 assignatures_obligatories.txt
37 assignatures leides de 3 graus diferents.

*********************************
* Torn 1                   n=5  *
*-------------------------------*
* 3105G03023 (tipus g),   GEB-3 *
* 3105G03013 (tipus g),   GEB-2 *
* 3105G13034 (tipus r),  GDDV-3 *
* 3105G07019 (tipus r), GEINF-3 *
* 3105G07017 (tipus r), GEINF-2 *
*********************************

*********************************
* Torn 2                   n=5  *
*-------------------------------*
* 3105G03012 (tipus g),   GEB-2 *
* 3105G07012 (tipus g), GEINF-3 *
* 3105G15008 (tipus r),   GEB-3 *
* 3105G15012 (tipus r),   GEB-4 *
* 3105G15001 (tipus r),   GEB-1 *
*********************************

*********************************
* Torn 3                   n=5  *
*-------------------------------*
* 3105G00015 (tipus g),   GEB-2 *
* 3105G07021 (tipus g), GEINF-2 *
* 3105G15009 (tipus r),   GEB-3 *
* 3105G15013 (tipus r),   GEB-4 *
* 3105G13024 (tipus r),  GDDV-4 *
*********************************

*********************************
* Torn 4                   n=5  *
*-------------------------------*
* 3105G00004 (tipus g),   GEB-1 *
* 3105G07004 (tipus g), GEINF-1 *
* 3105G13014 (tipus r),  GDDV-4 *
* 3105G07007 (tipus r), GEINF-2 *
* 3105G07026 (tipus r), GEINF-3 *
*********************************

*********************************
* Torn 5                   n=5  *
*-------------------------------*
* 3105G07002 (tipus g), GEINF-1 *
* 3105G07010 (tipus g), GEINF-2 *
* 3105G15014 (tipus r),   GEB-4 *
* 3105G07013 (tipus r), GEINF-3 *
* 3105G13030 (tipus r),  GDDV-3 *
*********************************

*********************************
* Torn 6                   n=5  *
*-------------------------------*
* 3105G07001 (tipus g), GEINF-1 *
* 3105G07027 (tipus g), GEINF-3 *
* 3105G13022 (tipus r),  GDDV-3 *
* 3105G07035 (tipus r), GEINF-4 *
* 3105G13029 (tipus r),  GDDV-2 *
*********************************

*********************************
* Torn 7                   n=5  *
*-------------------------------*
* 3105G00001 (tipus g),   GEB-1 *
* 3105G07023 (tipus g), GEINF-3 *
* 3105G13031 (tipus r),  GDDV-1 *
* 3105G15007 (tipus r),   GEB-3 *
* 3105G13025 (tipus r),  GDDV-4 *
*********************************

*********************************
* Torn 8                   n=2  *
*-------------------------------*
* 3105G07005 (tipus g), GEINF-1 *
* 3105G13001 (tipus r),  GDDV-1 *
*********************************

Num. torns: 8
Num. dies:  4
Desviaci¡ó¡: 1.31907
Temps: 0.00347749 segons
\end{lstlisting}


Amb l'algoritme voraç:
\begin{lstlisting}[style=codibash]
$ ./p2 -v -cr 3 -gc 2 assignatures_obligatories.txt
37 assignatures leides de 3 graus diferents.

*********************************
* Torn 1                   n=5  *
*-------------------------------*
* 3105G03023 (tipus g),   GEB-3 *
* 3105G03013 (tipus g),   GEB-2 *
* 3105G15012 (tipus r),   GEB-4 *
* 3105G07026 (tipus r), GEINF-3 *
* 3105G07035 (tipus r), GEINF-4 *
*********************************

*********************************
* Torn 2                   n=5  *
*-------------------------------*
* 3105G03012 (tipus g),   GEB-2 *
* 3105G07012 (tipus g), GEINF-3 *
* 3105G15009 (tipus r),   GEB-3 *
* 3105G15014 (tipus r),   GEB-4 *
* 3105G13029 (tipus r),  GDDV-2 *
*********************************

*********************************
* Torn 3                   n=5  *
*-------------------------------*
* 3105G00015 (tipus g),   GEB-2 *
* 3105G07021 (tipus g), GEINF-2 *
* 3105G15013 (tipus r),   GEB-4 *
* 3105G07013 (tipus r), GEINF-3 *
* 3105G15007 (tipus r),   GEB-3 *
*********************************

*********************************
* Torn 4                   n=4  *
*-------------------------------*
* 3105G00004 (tipus g),   GEB-1 *
* 3105G07004 (tipus g), GEINF-1 *
* 3105G13024 (tipus r),  GDDV-4 *
* 3105G13031 (tipus r),  GDDV-1 *
*********************************

*********************************
* Torn 5                   n=5  *
*-------------------------------*
* 3105G07002 (tipus g), GEINF-1 *
* 3105G07010 (tipus g), GEINF-2 *
* 3105G15001 (tipus r),   GEB-1 *
* 3105G13030 (tipus r),  GDDV-3 *
* 3105G13025 (tipus r),  GDDV-4 *
*********************************

*********************************
* Torn 6                   n=4  *
*-------------------------------*
* 3105G07001 (tipus g), GEINF-1 *
* 3105G07027 (tipus g), GEINF-3 *
* 3105G13014 (tipus r),  GDDV-4 *
* 3105G13022 (tipus r),  GDDV-3 *
*********************************

*********************************
* Torn 7                   n=4  *
*-------------------------------*
* 3105G00001 (tipus g),   GEB-1 *
* 3105G07023 (tipus g), GEINF-3 *
* 3105G07007 (tipus r), GEINF-2 *
* 3105G13001 (tipus r),  GDDV-1 *
*********************************

*********************************
* Torn 8                   n=5  *
*-------------------------------*
* 3105G07005 (tipus g), GEINF-1 *
* 3105G13034 (tipus r),  GDDV-3 *
* 3105G07019 (tipus r), GEINF-3 *
* 3105G15008 (tipus r),   GEB-3 *
* 3105G07017 (tipus r), GEINF-2 *
*********************************

Num. torns: 8
Num. dies:  4
Desviaci¡ó¡: 1.26381
Temps: 0.00221079 segons
\end{lstlisting}


I amb \textit{backtracking} de nou, però buscant l'òptim, l'arbre de possibilitats creix tant que ja és inviable.
%\begin{lstlisting}[style=codibash]
%$ p2 -v -cr 3 -gc 2 assignatures_obligatories.txt
%37 assignatures leides de 3 graus diferents.
%
%\end{lstlisting}



\end{document}